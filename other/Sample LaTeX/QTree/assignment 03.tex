% Nathan Schulte - nmschulte@unomaha.edu
% Copyright 2012 Nathan Schulte

\documentclass{article}

\textheight=9.1in
\textwidth=6.6in
\topmargin=0.0in
\headheight=0.0in
\headsep=0.0in
\oddsidemargin=-0.19in

\usepackage{amssymb}
\usepackage{amsmath}
\usepackage{tikz}
\usepackage{tikz-qtree}

\usetikzlibrary{arrows,automata,positioning}

\let\oldemptyset\emptyset
\let\emptyset\varnothing

\begin{document}

\title{Assignment 03}
\date{}
\author{Nathan Schulte \\ \texttt{nmschulte@unomaha.edu}}
\maketitle

%----------------
\subsection*{Problem 1.}
%----------------

\begin{align*}
  S &\rightarrow \texttt{1}\:|\:\texttt{0}\:|\:\texttt{1}A\texttt{1}\:|\:\texttt{0}A\texttt{0}\\
  A &\rightarrow \texttt{1}\:|\:\texttt{0}\:|\:\varepsilon\:|\:AA
\end{align*}


%----------------
\subsection*{Problem 2.}
%----------------

\begin{enumerate}
  \item Make a new start variable.\\
    \begin{align*}
      S &\rightarrow A\\
      A &\rightarrow BAB\:|\:B\:|\:\varepsilon\\
      B &\rightarrow \texttt{00}\:|\:\varepsilon
    \end{align*}
  \item Remove \(\varepsilon\)-rules.\\
    \begin{align*}
      S &\rightarrow A\:|\:\varepsilon\\
      A &\rightarrow BAB\:|\:B\:|\:BB\:|\:A\:|\:AB\:|\:BA\\
      B &\rightarrow \texttt{00}
    \end{align*}
  \item Remove unit rules.\\
    \begin{align*}
      S &\rightarrow BAB\:|\:\texttt{00}\:|\:BB\:|\:AB\:|\:BA\:|\:\varepsilon\\
      A &\rightarrow BAB\:|\:\texttt{00}\:|\:BB\:|\:AB\:|\:BA\\
      B &\rightarrow \texttt{00}
    \end{align*}
  \item Convert rules into proper form.\\
    \begin{align*}
      S &\rightarrow BU\:|\:VV\:|\:BB\:|\:AB\:|\:BA\:|\:\varepsilon\\
      A &\rightarrow BU\:|\:VV\:|\:BB\:|\:AB\:|\:BA\\
      B &\rightarrow VV\\
      U &\rightarrow AB\\
      V &\rightarrow \texttt{0}
    \end{align*}
\end{enumerate}


%----------------
\subsection*{Problem 3.}
%----------------

\(G = \left(\left\{S\right\}, \left\{\texttt{0}\right\}, R, S\right)\).  The set of rules, \(R\), is
  \begin{align*}
    S &\rightarrow \texttt{0}.
  \end{align*}
Adding the rule \(S \rightarrow SS\) yields the grammer \(G'\) with the set of rules
  \begin{align*}
    S &\rightarrow \texttt{0}\:|\:SS
  \end{align*}
\(G'\) does not generate \(A^*\), where \(A = L\left(G\right)\); it is missing the empty string, \(\lambda\).


%----------------
\subsection*{Problem 4.}
%----------------

\begin{enumerate}
  \item All strings, i.e. \(\Sigma^*\), of the alphabet \(\left\{\texttt{a}, \texttt{b}\right\}\), except those of the form \(\texttt{a}^n\texttt{b}^n\).
  \item \(G_2 = \left(\left\{S\right\}, \left\{\texttt{a}, \texttt{b}\right\}, R, S\right)\).  The set of rules, \(R\), is
    \begin{align*}
      S &\rightarrow \texttt{a}S\texttt{b}\:|\:\varepsilon.
    \end{align*}
\end{enumerate}


%----------------
\subsection*{Problem 5.}
%----------------

\begin{enumerate}
  \item The string \texttt{if condition then if condition then x := 0 else x := 0} is derived ambiguously by the grammer \(G\).
  \begin{center}
    \Tree [.STMT
            [.IF\_THEN
              [.\texttt{if} ]
              [.\texttt{condition} ]
              [.\texttt{then} ]
              [.STMT
                [.IF\_THEN\_ELSE
                  [.\texttt{if} ]
                  [.\texttt{condition} ]
                  [.\texttt{then} ]
                  [.STMT
                    [.ASSIGN
                      [.\texttt{x} ]
                      [.\texttt{:=} ]
                      [.\texttt{0} ] ] ]
                 [.\texttt{else} ]
                 [.STMT
                   [.ASSIGN
                     [.\texttt{x} ]
                     [.\texttt{:=} ]
                     [.\texttt{0} ] ] ] ] ] ] ] \\
    \Tree [.STMT
            [.IF\_THEN\_ELSE
              [.\texttt{if} ]
              [.\texttt{condition} ]
              [.\texttt{then} ]
              [.STMT
                [.IF\_THEN
                  [.\texttt{if} ]
                  [.\texttt{condition} ]
                  [.\texttt{then} ]
                  [.STMT
                    [.ASSIGN
                      [.\texttt{x} ]
                      [.\texttt{:=} ]
                      [.\texttt{0} ] ] ] ] ]
              [.\texttt{else} ]
              [.STMT
                [.ASSIGN
                  [.\texttt{x} ]
                  [.\texttt{:=} ]
                  [.\texttt{0} ] ] ] ] ]
  \end{center}
  \item \(G = \left(V, \Sigma, R, \mathrm{STMT}\right)\), where
  \begin{enumerate}
    \item \(V = \left\{\mathrm{STMT}, \mathrm{ASSIGN}, \mathrm{MATCHED}, \mathrm{UNMATCHED}\right\}\)
    \item \(\Sigma = \left\{\texttt{if}, \texttt{condition}, \texttt{then}, \texttt{else}, \texttt{x}, \texttt{:=}, \texttt{0}\right\}\)
    \item the set of rules, \(R\), is
    \begin{align*}
      \mathrm{STMT} \rightarrow&\:\mathrm{ASSIGN}\\
      |&\:\mathrm{MATCHED}\\
      |&\:\mathrm{UNMATCHED}\\
      \mathrm{MATCHED} \rightarrow&\:\texttt{if condition then }\mathrm{MATCHED} \texttt{ else } \mathrm{MATCHED}\\
      |&\:\mathrm{ASSIGN}\\
      \mathrm{UNMATCHED} \rightarrow&\:\texttt{if condition then } \mathrm{STMT}\\
      |&\:\texttt{if condition then } \mathrm{MATCHED} \texttt{ else } \mathrm{STMT}\\
      \mathrm{ASSIGN} \rightarrow&\: \texttt{x := 0}
    \end{align*}
  \end{enumerate}
\end{enumerate}


%----------------
\subsection*{Problem 6.}
%----------------

\begin{enumerate}
  \item \(L\left(G\right) = L\)
  \item Let \(k\) be the pumping lemma length given by the pumping lemma.
  \item \(z = \texttt{0}^{k}\texttt{1}\texttt{0}^{2k}\texttt{1}\texttt{0}^{3k} \in L\)
  \item \(z = uvwxy \wedge vx \neq \lambda \wedge \left|vwx\right| \leq k\)
  \item Case 1: \(vx\) is all \texttt{0}s before the first \texttt{1}.  In this case, the relationship between the sets of \texttt{0}s does not hold; the ratios are no longer \(1:2:3\).\\
  Case 2: \(vx\) contains the first \texttt{1}.  In this case, when pumping up or down, the number of \texttt{1}s is no longer two.\\
  Case 3: \(vx\) is all \texttt{0}s from the middle set of \texttt{0}s.  In this case, the relationship between the sets of \texttt{0}s does not hold; the ratios are no longer \(1:2:3\).\\
  Case 4: \(vx\) contains the second \texttt{1}.  In this case, when pumping up or down, the number of \texttt{1}s is no longer two.\\
  Case 5: \(vx\) is all \texttt{0}s from the after the last \texttt{1}.  In this case, the relationship between the sets of \texttt{0}s does not hold; the ratios are no longer \(1:2:3\).\\
\end{enumerate}


\end{document}
